\documentclass[compress]{beamer}

\usetheme{Hamburg}

\usepackage[T1]{fontenc}
\usepackage[utf8]{inputenc}

\usepackage{lmodern}

%\usepackage[english]{babel}
\usepackage[ngerman]{babel}

\usepackage{eurosym}
\usepackage{listings}
\usepackage{lstautogobble}
\usepackage{microtype}
\usepackage{textcomp}
\usepackage{siunitx}
\usepackage{csquotes}
\usepackage{graphicx}
\usepackage{xcolor}
\usepackage{tikz}
\usepackage{tabularx}
\usepackage{physics}
\usepackage{amsmath}
\usepackage{amsfonts}
\usepackage{amssymb}
\usepackage{mathtools}
\usepackage{todonotes}
\usepackage{xfrac}
\usepackage{multirow}
\usepackage{hyperref}
\usepackage[noabbrev]{cleveref}

\sisetup{%
    locale = DE,
    per-mode = symbol,
    range-phrase = \text{--},
    range-units = single,
    separate-uncertainty = true
}

\lstset{
    basicstyle=\ttfamily\footnotesize,
    frame=single,
    numbers=left,
    language=C,
    breaklines=true,
    breakatwhitespace=true,
    postbreak=\hbox{$\hookrightarrow$ },
    showstringspaces=false,
    autogobble=true,
    upquote=true,
    tabsize=4,
    captionpos=b,
    morekeywords={int8_t,uint8_t,int16_t,uint16_t,int32_t,uint32_t,int64_t,uint64_t,size_t,ssize_t,off_t,intptr_t,uintptr_t,mode_t}
}

\title{HowTo: Linux}
% \subtitle{Veranstaltung}
\author{Ruben 14felgenh, Hauke 14stieler}
%\institute{Arbeitsbereich Wissenschaftliches Rechnen\\Fachbereich Informatik\\Fakultät für Mathematik, Informatik und Naturwissenschaften\\Universität Hamburg}

\usepackage{datetime}
\newcommand{\datef}[3]{%
    \newdate{datex}{#1}{#2}{#3}
    \date{\protect\displaydate{datex}}
}
\let\oldthedate\thedate
\makeatletter
\makeatother
\datef{06}{12}{2022}

%\usepackage[style=numeric,bibencoding=utf8,backend=biber]{biblatex}
%\addbibresource{literature.bib}

\begin{document}

    \begin{frame}
        \titlepage
    \end{frame}

\section{Basics}

\begin{frame}[fragile]{Begriffe}
\begin{itemize}
    \item Kommandozeile / Konsole / CLI
    \item Shell
    \begin{itemize}
\item \verb+sh+
\item \verb+bash+
\item \verb+zsh+
\item \verb+ksh+
\item \verb+csh+
\item \verb+dash+
\item \verb+fish+
\end{itemize}
\item Terminal
\item Terminal-Emulator
\end{itemize}
\end{frame}

\begin{frame}[fragile]{Befehle}
\begin{itemize}
\item Shell-Built-Ins:
\begin{itemize}
\item \verb+pwd+
\item \verb+cd+
\item \verb+echo+
\item \verb+true+
\item \verb+false+
\item \verb+type+
\end{itemize}
\item Programme:
\begin{itemize}
\item \verb+firefox+
\item \verb+touch+
\item \verb+ls+
\item \verb+cat+
\item \verb+man+
\item \verb+less+
\end{itemize}
\end{itemize}
\end{frame}

\begin{frame}[fragile]{Argumente u.Ä.}
Ein paar Beispiele für Flags, Parameter und Argumente:
\begin{itemize}
\item \verb+type echo+
\item \verb+ls --color=always+
\item \verb+dd if=/dev/urandom count=1+
\item \verb+firefox --help+
\item \verb+firefox -h+
\item \verb+man echo+
\end{itemize}
\end{frame}

\begin{frame}[fragile]{Ein- und Ausgaben}
\begin{itemize}
\item \verb+stdout+
\item \verb+stderr+
\item \verb+stdin+
\item Exit Codes
\item Ausgaben in Dateien umleiten:\\
\verb+echo foobar > foobar.txt+
\item Ausgaben als Eingaben für andere Programme verwenden: \\
\verb+echo foobar | rev+
\item Ausgaben in Variablen speichern:\\
\verb+meinname=$(whoami)+\\
\verb+echo $meinname+
\end{itemize}
\end{frame}

\begin{frame}[fragile]{History und Autovervollständigung}
\begin{itemize}
\item History:
\begin{itemize}
\item Pfeiltasten
\item \verb+history+
\item \verb+cat ~/.bash_history+
\item Strg + R
\end{itemize}
\item Autovervollständigung: Tab
\end{itemize}
\end{frame}

\section{Dateisystem}

\begin{frame}[fragile]{Dateien}
\begin{itemize}
\item Erstellen: \verb+touch foo.txt+
\item Lesen: \verb+cat foo.txt+ oder \verb+less foo.txt+
\item Schreiben: Benutze \verb+>+ oder \verb+>>+
\item Filtern: \verb+grep suchtext foo.text+
\item Kopieren: \verb+cp foo.txt bar.txt+
\item Umbenennen / Verschieben: \verb+mv foo.txt foobar.txt+
\item Löschen: \verb+rm bar.txt+
\end{itemize}
\end{frame}

\begin{frame}[fragile]{Ordner}
\begin{itemize}
\item Erstellen: \verb+mkdir foo+
\item Inhalt auflisten: \verb+ls+
\item Kopieren: \verb+cp -r foo bar+
\item Umbenennen / Verschieben: \verb+mv foo foobar+
\item Löschen: \verb+rm -r bar+
\end{itemize}
\end{frame}

\begin{frame}[fragile]{Dateien und Ordner suchen}
    content...
\end{frame}

\begin{frame}[fragile]{Berechtigungen}
foo
\end{frame}

\begin{frame}[fragile]{Besitzer}
    content...
\end{frame}

\section{Netzwerk und Internet}

\begin{frame}[fragile]{Dateien herunterladen}
    content...
\end{frame}

\subsection{Dateien Herunter- \& Hochladen}

\subsection{SSH}

\subsubsection{SSH-Keys Nutzen}

\subsubsection{SCP}

\section{Prozesse}

\subsection{Prozesse Auflisten und System-Auslastung Anschauen}

\subsection{Prozess Beenden}

\section{Texteditoren}

\subsection{nano}

\subsection{micro}

\subsection{vim}

\section{Konfiguration der Bash}

\section{man / help}

\section{Shell-Skripte}

\end{document}
